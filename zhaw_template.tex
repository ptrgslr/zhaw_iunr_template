\documentclass[a4paper,12pt,twoside]{article}


\usepackage[utf8]{inputenc}
\usepackage[ngerman]{babel}
\usepackage{graphicx}
\usepackage{geometry}
\usepackage{setspace}
\usepackage{fancyhdr}
\usepackage{titlesec}
\usepackage{tocbibind}
\usepackage{csquotes}
\usepackage[backend=biber,style=authoryear]{biblatex}
\usepackage{hyperref}
\usepackage{helvet}
\renewcommand{\familydefault}{\sfdefault} % Serifenlose Schrift wie Arial
\usepackage[utf8]{inputenc}
\usepackage{caption}
\usepackage{tocloft} % Für Inhaltsverzeichnis-Anpassungen

% Seitenränder für doppelseitiges Layout
\geometry{top=2.5cm, bottom=2.5cm, left=2.5cm, right=2.5cm}

% Zeilenabstand
\onehalfspacing

% Kopf- und Fusszeilen mit fancyhdr
\fancypagestyle{standard}{
	\fancyhf{} % Leert die Kopf- und Fusszeilen
	% Kopfzeilen
	\fancyhead[LE]{\nouppercase{\leftmark}} % Kapitelname links auf geraden Seiten (kein ALL CAPS)
	\fancyhead[RO]{\nouppercase{\leftmark}} % Kapitelname rechts auf ungeraden Seiten (kein ALL CAPS)
	% Fusszeilen
	\fancyfoot[LE]{\thepage}                % Seitenzahl links auf geraden Seiten
	\fancyfoot[RO]{\thepage}                % Seitenzahl rechts auf ungeraden Seiten
}

\fancypagestyle{appendix}{
	\fancyhf{}
	% Kopfzeilen
	\fancyhead[LE]{Anhang}                  % "Anhang" links auf geraden Seiten
	\fancyhead[RO]{Anhang}                  % "Anhang" rechts auf ungeraden Seiten
	% Fuzeilen
	\fancyfoot[LE]{\thepage}                % Seitenzahl links auf geraden Seiten
	\fancyfoot[RO]{\thepage}                % Seitenzahl rechts auf ungeraden Seiten
}

\fancypagestyle{plain}{%
	\fancyhf{} % Leert die Kopf- und Fuzeilen
	\fancyhead[LE]{\nouppercase{\leftmark}} % Kapitelname links auf geraden Seiten
	\fancyhead[RO]{\nouppercase{\leftmark}} % Kapitelname rechts auf ungeraden Seiten
	\fancyfoot[LE]{\thepage} % Seitenzahl links auf geraden Seiten
	\fancyfoot[RO]{\thepage} % Seitenzahl rechts auf ungeraden Seiten
}

% Inhaltsverzeichnis anpassen
\renewcommand{\cfttoctitlefont}{\large\bfseries} % Titel Inhaltsverzeichnis in \large und fett
\renewcommand{\cftsecfont}{\normalfont} % Sektionen nicht fett
\renewcommand{\cftsubsecfont}{\normalfont} % Subsektionen nicht fett
\renewcommand{\cftsecpagefont}{\normalfont} % Seitenzahlen nicht fett
\renewcommand{\cftsubsecpagefont}{\normalfont} % Seitenzahlen von Subsektionen nicht fett

% Punkte für alle Ebenen im Inhaltsverzeichnis
\renewcommand{\cftsecleader}{\cftdotfill{\cftdotsep}} % Punkte für \section
\renewcommand{\cftsubsecleader}{\cftdotfill{\cftdotsep}} % Punkte für \subsection
\renewcommand{\cftsubsubsecleader}{\cftdotfill{\cftdotsep}} % Punkte für \subsubsection

% Einzug für jede Ebene anpassen
\setlength{\cftsecindent}{0em} % Keine Einrückung für \section
\setlength{\cftsubsecindent}{2em} % Einrückung für \subsection
\setlength{\cftsubsubsecindent}{4em} % Einrückung für \subsubsection

% Zeigt Subsections im Inhaltsverzeichnis an
\setcounter{tocdepth}{2}
% Optional: Nummeriert Subsections
\setcounter{secnumdepth}{2}

% Keine Kopf- und Fusszeilen vor der Einleitung
\pagestyle{empty} % "empty" deaktiviert die Kopf- und Fusszeilen

% Überschriftenformatierung
\titleformat{\section}{\normalfont\large\bfseries}{\thesection}{1em}{}
\titleformat{\subsection}{\normalfont\normalsize\bfseries}{\thesubsection}{1em}{}
\titleformat{\subsubsection}{\normalfont\normalsize\bfseries}{\thesubsubsection}{1em}{}

% Globale Einstellung für Tabellen-Captions
\captionsetup[table]{justification=raggedright, singlelinecheck=false}
\captionsetup{{labelfont=bf, small}, textfont=small}
% Tabellen-/Abbildungsbeschriftung kleiner und fettgedruckt, text nur small

% Hyperlinks
\hypersetup{
	colorlinks=true,
	linkcolor=black,
	citecolor=black,
	filecolor=black,
	urlcolor=blue,
	pdftitle={Titel der Arbeit},
	pdfauthor={Autor/in},
	pdfsubject={Art der Arbeit},
	pdfkeywords={Schlagworte}
}



% BibLaTeX-Bibliografie-Datei einbinden
\addbibresource{literatur.bib} % Pfad zur BibTeX-Datei

\begin{document}
	
	% Titelseite
	\begin{titlepage}
		\centering
		{\Large Zürcher Hochschule für Angewandte Wissenschaften\\
			Departement Life Sciences und Facility Management\\[2cm]}
		{\Large \textbf{Titel der Arbeit}\\[1cm]}
		{\large LaTex Vorlage\\[1cm]}
		{\large \textbf{Art der Arbeit}\\[1cm]}
		\begin{tabbing}
			\hspace*{6cm} \= \kill
			\textbf{Autor/in:}    \> Dein Name\\
			\textbf{Studiengang:} \> Dein Studiengang\\
			\textbf{Betreuer/in:} \> Name des/der Betreuers/in\\
			\textbf{Abgabedatum:} \> TT.MM.JJJJ
		\end{tabbing}
		\vfill
	\end{titlepage}
	
	% Impressum auf der Rückseite des Titelblatts
	\newpage
	\thispagestyle{empty}
	\noindent
	
	\textbf{Schlagworte:} Schlagwort1, Schlagwort2, Schlagwort3\\[1cm]
	\textbf{Zitiervorschlag:}\\
	Autor/in (Jahr). Titel der Arbeit. Zürcher Hochschule für Angewandte Wissenschaften, Departement Life Sciences und Facility Management.\\[1cm]
	\textbf{Institut:} Name des Instituts
	
	% Zusammenfassung
	\newpage
	\section*{Zusammenfassung}
	Hier folgt die Zusammenfassung der Arbeit.
	
	% Abkürzungsverzeichnis
	\newpage
	\section*{Abkürzungsverzeichnis}
	\begin{tabbing}
		\hspace*{3cm} \= \kill
		Abb. \> Abbildung\\
		Tab. \> Tabelle\\
		etc.
	\end{tabbing}
	
	
	% Inhaltsverzeichnis
	\newpage
	\addtocontents{toc}{\protect\setcounter{tocdepth}{-1}} %kapitel Inhaltsverzeichnis ausschliessen
	\tableofcontents
	\addtocontents{toc}{\protect\setcounter{tocdepth}{3}}
	\thispagestyle{empty}
	
	
	% Einleitung
	\newpage
	\pagestyle{standard} % Aktiviert Kopf- und Fußzeilen ab dieser Seite
	\section{Einleitung}
	\addcontentsline{toc}{section}{Einleitung} % Fügt "Einleitung" in das Inhaltsverzeichnis ein
	Hier beginnt der Hauptteil der Arbeit.
	\subsection{Stand der Forschung}
	Hier beginnt der Hauptteil der Arbeit.
	
	
	% Weitere Kapitel
	\newpage
	\section{Theorie oder Literaturübersicht}
	% Inhalt
	
	\newpage
	\section{Material und Methoden}
	% Inhalt
	
	\newpage
	\section{Ergebnisse}
	% Inhalt
	
	\newpage
	\section{Diskussion}
	% Inhalt
	
	% Literaturverweise im Text
	% Beispiel für ein Zitat:
	% \cite{SchlüsselAusBibDatei}
	
	% Literaturverzeichnis
	\newpage
	\pagestyle{plain} % Setzt den neuen Seitenstil
	\printbibliography
	\markboth{Literaturverzeichnis}{Literaturverzeichnis} % Kapitelname in der Kopfzeile
	
	
	% Abbildungsverzeichnis (falls benötigt)
	\newpage
	\pagestyle{plain} % Setzt den neuen Seitenstil
	\listoffigures
	\markboth{Abbildungsverzeichnis}{Abbildungsverzeichnis} % Kapitelname in der Kopfzeile
	
	
	% Tabellenverzeichnis (falls benötigt)
	\newpage
	\pagestyle{plain} % Setzt den neuen Seitenstil
	\listoftables
	\markboth{Tabellenverzeichnis}{Tabellenverzeichnis} % Kapitelname in der Kopfzeile
	
	
	% Anhang (falls benötigt)
	\newpage
	\appendix
	\pagenumbering{arabic} % Seitenzahlen im Anhang beginnen bei 1
	\pagestyle{appendix}   % Seitenstil für den Anhang (gerade/ungerade)
	\section*{Anhang}
	\addcontentsline{toc}{section}{Anhang}
	\renewcommand{\thesubsection}{Anhang \Alph{subsection}}
	% Inhalt
	
	
\end{document}
