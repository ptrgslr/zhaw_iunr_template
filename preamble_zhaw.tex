
\usepackage[utf8]{inputenc}
\usepackage[ngerman]{babel}
\usepackage{graphicx}
\usepackage{geometry}
\usepackage{setspace}
\usepackage{fancyhdr}
\usepackage{titlesec}
\usepackage{tocbibind}
\usepackage{csquotes}
\usepackage[backend=biber,style=authoryear]{biblatex}
\usepackage{hyperref}
\usepackage{helvet}
\renewcommand{\familydefault}{\sfdefault} % Serifenlose Schrift wie Arial

% Seitenränder für doppelseitiges Layout
\geometry{top=2.5cm, bottom=2.5cm, left=2.5cm, right=2.5cm}

% Zeilenabstand
\onehalfspacing

% Kopf- und Fußzeilen mit fancyhdr
\fancypagestyle{standard}{
	\fancyhf{} % Leert die Kopf- und Fußzeilen
	% Kopfzeilen
	\fancyhead[LE]{\nouppercase{\leftmark}} % Kapitelname links auf geraden Seiten (kein ALL CAPS)
	\fancyhead[RO]{\nouppercase{\leftmark}} % Kapitelname rechts auf ungeraden Seiten (kein ALL CAPS)
	% Fußzeilen
	\fancyfoot[LE]{\thepage}                % Seitenzahl links auf geraden Seiten
	\fancyfoot[RO]{\thepage}                % Seitenzahl rechts auf ungeraden Seiten
}

\fancypagestyle{appendix}{
	\fancyhf{}
	% Kopfzeilen
	\fancyhead[LE]{Anhang}                  % "Anhang" links auf geraden Seiten
	\fancyhead[RO]{Anhang}                  % "Anhang" rechts auf ungeraden Seiten
	% Fußzeilen
	\fancyfoot[LE]{\thepage}                % Seitenzahl links auf geraden Seiten
	\fancyfoot[RO]{\thepage}                % Seitenzahl rechts auf ungeraden Seiten
}

\fancypagestyle{plain}{%
	\fancyhf{} % Leert die Kopf- und Fußzeilen
	\fancyhead[LE]{\nouppercase{\leftmark}} % Kapitelname links auf geraden Seiten
	\fancyhead[RO]{\nouppercase{\leftmark}} % Kapitelname rechts auf ungeraden Seiten
	\fancyfoot[LE]{\thepage} % Seitenzahl links auf geraden Seiten
	\fancyfoot[RO]{\thepage} % Seitenzahl rechts auf ungeraden Seiten
}

% Inhaltsverzeichnis anpassen
\usepackage{tocloft} % Für Inhaltsverzeichnis-Anpassungen
\renewcommand{\cftsecfont}{\normalfont} % Sektionen nicht fett
\renewcommand{\cftsubsecfont}{\normalfont} % Subsektionen nicht fett
\renewcommand{\cftsecpagefont}{\normalfont} % Seitenzahlen nicht fett
\renewcommand{\cftsubsecpagefont}{\normalfont} % Seitenzahlen von Subsektionen nicht fett

% Punkte für alle Ebenen im Inhaltsverzeichnis
\renewcommand{\cftsecleader}{\cftdotfill{\cftdotsep}} % Punkte für \section
\renewcommand{\cftsubsecleader}{\cftdotfill{\cftdotsep}} % Punkte für \subsection
\renewcommand{\cftsubsubsecleader}{\cftdotfill{\cftdotsep}} % Punkte für \subsubsection

% Einzug für jede Ebene anpassen
\setlength{\cftsecindent}{0em} % Keine Einrückung für \section
\setlength{\cftsubsecindent}{2em} % Einrückung für \subsection
\setlength{\cftsubsubsecindent}{4em} % Einrückung für \subsubsection

% Zeigt Subsections im Inhaltsverzeichnis an
\setcounter{tocdepth}{2}
% Optional: Nummeriert Subsections
\setcounter{secnumdepth}{2}

% Keine Kopf- und Fußzeilen vor der Einleitung
\pagestyle{empty} % "empty" deaktiviert die Kopf- und Fusszeilen

% Überschriftenformatierung
\titleformat{\section}{\normalfont\Large\bfseries}{\thesection}{1em}{}
\titleformat{\subsection}{\normalfont\large\bfseries}{\thesubsection}{1em}{}
\titleformat{\subsubsection}{\normalfont\normalsize\bfseries}{\thesubsubsection}{1em}{}



% Hyperlinks
\hypersetup{
	colorlinks=true,
	linkcolor=black,
	citecolor=black,
	filecolor=black,
	urlcolor=blue,
	pdftitle={Titel der Arbeit},
	pdfauthor={Autor/in},
	pdfsubject={Art der Arbeit},
	pdfkeywords={Schlagworte}
}

%----------------------------
%   Fonts and characters
%----------------------------

% Support for special characters
\usepackage[utf8]{inputenc}    % Specify input encoding
\usepackage[T1]{fontenc}       % Specify font encoding

% Set main fonts
% Fonts catalogue: https://tug.org/FontCatalogue/
\usepackage{mathpazo}          % Use the Palatino font by default
\usepackage{beramono}          % Override the monospace/typewriter font

% ZHAW title font
% Try to load Helvetica Rounded Bold, and OpenType font.
% Loading OTF or system fonts is possible with XeLaTeX.
% If the document is compiled using pdfLaTeX, resort 
\usepackage{ifxetex}
\ifxetex
\usepackage{fontspec}
\newfontfamily\zhawtitlefont{Helvetica Rounded Bold}
\else
\newcommand{\zhawtitlefont}{\scshape}
\fi

%\usepackage[scaled]{helvet}

%----------------------------
%   Environments
%----------------------------

\usepackage{caption}           % Customized caption
\usepackage{subcaption}        % Subfigure captions
\usepackage{makecell}          % Per-cell formatting in tables (\makecell)
\usepackage{pdfpages}          % Required to include PDF files/graphics (\includepdf)

\usepackage{todonotes}         % Introduces the command \todo
\setlength{\marginparwidth}{2.5cm} % Adjust this if the todo notes are out of margins

% Create boxes as follows:
% \begin{colorbox}{red}{2}
	\usepackage{tcolorbox}
	\newtcolorbox{textbox}[2]{
		arc=3pt,
		boxrule=#2pt,
		colback=#1!25!white,
		width=\textwidth,
		halign=left,
		valign=center,
		colframe=#1!75!black
	}
	
	%----------------------------
	%   Colors
	%----------------------------
	
	% Set up colors
	\usepackage{xcolor}
	% ZHAW Blue: Pantone 2945 U / R0 G100 B166
	\definecolor{zhawblue}{rgb}{0.00, 0.39, 0.65}
	% Colors related to code listings
	\definecolor{codegreen}{rgb}{0,0.6,0}
	\definecolor{codegray}{rgb}{0.5,0.5,0.5}
	\definecolor{codepurple}{rgb}{0.58,0,0.82}
	\definecolor{codebackground}{rgb}{0.93,0.94,0.95}
	
	%----------------------------
	%   Code listings
	%----------------------------
	
	% Setup code listings
	\usepackage{listings}
	\lstdefinestyle{mystyle}{
		backgroundcolor=\color{codebackground},   
		commentstyle=\color{codegreen},
		keywordstyle=\color{magenta},
		numberstyle=\tiny\color{codegray},
		stringstyle=\color{codepurple},
		basicstyle=\ttfamily\footnotesize,
		breakatwhitespace=false,
		breaklines=true,
		%    captionpos=b,
		keepspaces=true,
		numbers=left,
		numbersep=5pt,
		showspaces=false,
		showstringspaces=false,
		showtabs=false,
		tabsize=4
	}
	\lstset{style=mystyle}
	
	% minted is an alternative code listing package. (See chapter 2)
	% For it to run successfully, ensure the following:
	% - the Python package Pygments. Install with the following command:
	%       python -m pip install Pygments
	% - pdflatex (or xelatex) is executed with the flag --shell-escape
	%   If you are using a TEX editor, you can modify the typesetting 
	%   command somewhere in the settings.
	%\usepackage[outputdir=build]{minted}
	%\usemintedstyle{xcode}
	% For fancier coloring schemes, see here:
	% https://tex.stackexchange.com/questions/585582
	% One could also create an own style in Pygments
	% https://pygments.org/docs/styles/#creating-own-styles


% BibLaTeX-Bibliografie-Datei einbinden
\addbibresource{literatur.bib} % Pfad zur BibTeX-Datei