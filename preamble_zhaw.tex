
\usepackage[utf8]{inputenc}
\usepackage[ngerman]{babel}
\usepackage{graphicx}
\usepackage{geometry}
\usepackage{setspace}
\usepackage{fancyhdr}
\usepackage{titlesec}
\usepackage{tocbibind}
\usepackage{csquotes}
\usepackage[backend=biber,style=authoryear]{biblatex}
\usepackage{hyperref}
\usepackage{helvet}
\renewcommand{\familydefault}{\sfdefault} % Serifenlose Schrift wie Arial
\usepackage[utf8]{inputenc}
\usepackage{caption}
\usepackage{tocloft} % Für Inhaltsverzeichnis-Anpassungen

% Seitenränder für doppelseitiges Layout
\geometry{top=2.5cm, bottom=2.5cm, left=2.5cm, right=2.5cm}

% Zeilenabstand
\onehalfspacing

% Kopf- und Fusszeilen mit fancyhdr
\fancypagestyle{standard}{
	\fancyhf{} % Leert die Kopf- und Fusszeilen
	% Kopfzeilen
	\fancyhead[LE]{\nouppercase{\leftmark}} % Kapitelname links auf geraden Seiten (kein ALL CAPS)
	\fancyhead[RO]{\nouppercase{\leftmark}} % Kapitelname rechts auf ungeraden Seiten (kein ALL CAPS)
	% Fusszeilen
	\fancyfoot[LE]{\thepage}                % Seitenzahl links auf geraden Seiten
	\fancyfoot[RO]{\thepage}                % Seitenzahl rechts auf ungeraden Seiten
}

\fancypagestyle{appendix}{
	\fancyhf{}
	% Kopfzeilen
	\fancyhead[LE]{Anhang}                  % "Anhang" links auf geraden Seiten
	\fancyhead[RO]{Anhang}                  % "Anhang" rechts auf ungeraden Seiten
	% Fuzeilen
	\fancyfoot[LE]{\thepage}                % Seitenzahl links auf geraden Seiten
	\fancyfoot[RO]{\thepage}                % Seitenzahl rechts auf ungeraden Seiten
}

\fancypagestyle{plain}{%
	\fancyhf{} % Leert die Kopf- und Fuzeilen
	\fancyhead[LE]{\nouppercase{\leftmark}} % Kapitelname links auf geraden Seiten
	\fancyhead[RO]{\nouppercase{\leftmark}} % Kapitelname rechts auf ungeraden Seiten
	\fancyfoot[LE]{\thepage} % Seitenzahl links auf geraden Seiten
	\fancyfoot[RO]{\thepage} % Seitenzahl rechts auf ungeraden Seiten
}

% Inhaltsverzeichnis anpassen
\renewcommand{\cfttoctitlefont}{\Large\bfseries} % Titel Inhaltsverzeichnis in \large und fett
\renewcommand{\cftsecfont}{\large\bfseries} % Sektionen fett
\renewcommand{\cftsubsecfont}{\normalfont} % Subsektionen nicht fett
\renewcommand{\cftsecpagefont}{\large\bfseries} % Seitenzahlen fett
\renewcommand{\cftsubsecpagefont}{\normalfont} % Seitenzahlen von Subsektionen nicht fett

% Punkte für alle Ebenen im Inhaltsverzeichnis
\renewcommand{\cftsecleader}{\cftdotfill{\cftdotsep}} % Punkte für \section
\renewcommand{\cftsubsecleader}{\cftdotfill{\cftdotsep}} % Punkte für \subsection
\renewcommand{\cftsubsubsecleader}{\cftdotfill{\cftdotsep}} % Punkte für \subsubsection

% Einzug für jede Ebene anpassen
\setlength{\cftsecindent}{0em} % Keine Einrückung für \section
\setlength{\cftsubsecindent}{2em} % Einrückung für \subsection
\setlength{\cftsubsubsecindent}{4em} % Einrückung für \subsubsection

% Zeigt Subsections im Inhaltsverzeichnis an
\setcounter{tocdepth}{2}
% Optional: Nummeriert Subsections
\setcounter{secnumdepth}{3} %Subsubsections nummeriert mit %{3}%

% Keine Kopf- und Fusszeilen vor der Einleitung
\pagestyle{empty} % "empty" deaktiviert die Kopf- und Fusszeilen

% Überschriftenformatierung
\titleformat{\section}{\normalfont\Large\bfseries}{\thesection}{1em}{}
\titleformat{\subsection}{\normalfont\normalsize\bfseries}{\thesubsection}{1em}{}
\titleformat{\subsubsection}{\normalfont\normalsize\bfseries}{\thesubsubsection}{1em}{}

% Globale Einstellung für Tabellen-Captions
\captionsetup[table]{justification=raggedright, singlelinecheck=false}
\captionsetup{{labelfont=bf, small}, textfont=small}
% Tabellen-/Abbildungsbeschriftung kleiner und fettgedruckt, text nur small

% Hyperlinks
\hypersetup{
	colorlinks=true,
	linkcolor=black,
	citecolor=black,
	filecolor=black,
	urlcolor=blue,
	pdftitle={Titel der Arbeit},
	pdfauthor={Autor/in},
	pdfsubject={Art der Arbeit},
	pdfkeywords={Schlagworte}
}



% BibLaTeX-Bibliografie-Datei einbinden
\addbibresource{literatur.bib} % Pfad zur BibTeX-Datei